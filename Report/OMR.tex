\documentclass[a4paper,10pt]{article}
\usepackage[utf8]{inputenc}
\usepackage{listings}
\usepackage{graphicx}
%\newdate{date}{17}{02}{2017}
%\date{\displaydate{date}}

\title{Bubble sheet multiple choice scanner and test grader using computer vision and image processing techniques, along with Matlab\textsuperscript{\textregistered} Image Processing Toolbox}
\author{Marlen Akimaliev\\makimaliev@gmail.com}

\begin{document}
\maketitle
\section{Abstract}
\section{Introduction}
\section{What is Optical Mark Recognition (OMR)?}
Optical Mark Recognition, or OMR for short, is the process of automatically analyzing human-marked documents and interpreting their results. Arguably, the most famous, easily recognizable form of OMR are bubble sheet multiple choice tests, not unlike the ones you took in elementary school, middle school, or even high school. If you’re unfamiliar with “bubble sheet tests” or the trademark/corporate name of “Scantron tests”, they are simply multiple-choice tests that you take as a student. Each question on the exam is a multiple choice — and you use a \#2 pencil to mark the “bubble” that corresponds to the correct answer. 
The most notable bubble sheet test you experienced (at least in the United States) were taking the SATs during high school, prior to filling out college admission applications. I believe that the SATs use the software provided by Scantron to perform OMR and grade student exams, but I could easily be wrong there. In short, what I’m trying to say is that there is a massive market for Optical Mark Recognition and the ability to grade and interpret human-marked forms and exams.
\medskip
\begin{thebibliography}{9}
\bibitem{rosettacode} Rosetta Code Link, \\\texttt{https://rosettacode.org/wiki/Runge-Kutta-method}
\bibitem{euler} Euler Method PDF, \\\texttt{https://sites.math.washington.edu/~wcasper/math307-win16/review/euler-method/euler-method.pdf}
\bibitem{connor} Numerical Solutions to ODEs, \\\texttt{http://connor-johnson.com/2014/02/21/numerical-solutions-to-odes/}
\end{thebibliography}

\end{document}
